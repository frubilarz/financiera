% Template:     Informe/Reporte LaTeX
% Documento:    Archivo de ejemplo
% Versión:      5.5.4 (16/09/2018)
% Codificación: UTF-8
%
% Autor: Pablo Pizarro R. @ppizarror
%        Facultad de Ciencias Físicas y Matemáticas
%        Universidad de Chile
%        pablo.pizarro@ing.uchile.cl, ppizarror.com
%
% Manual template: [http://latex.ppizarror.com/Template-Informe/]
% Licencia MIT:    [https://opensource.org/licenses/MIT/]

\section{Introdicción}
Día a día el parque automotriz aumenta y es necesario mantener y tener nuevas señaléticas. Debido a esta necesidad es que nace Serviplott, una empresa la cual se encarga de vender servicios a distintos municipios los cuales requieren nuevas señaléticas y mantención de las señales viales ya existentes para así mejorar la experiencia de conducción.
\vspace{5mm}
Toda municipalidad tiene como tarea manejar las señales viales existentes en su sector. Al tener un número alto de señales en los distintos sectores es que las municipalidades optan por externalizar este servicio y se lo dejan a empresas importantes tales como Serviplott.
\vspace{5mm}
En el presente trabajo se desarrollará un completo análisis a la gestión de los procesos Industriales de Serviplott, identificando todos los problemas asociados a esta área, para luego proponer un plan de mejora en el ámbito de la Gestión de Procesos.
\vspace{5mm}
\insertimage[]{serviplott/logo2}{width=10cm}{Logo Empresa.}\newpage
\section{Objetivos}
	\subsection{Objetivos Generales}
	\subsection{Objetivos Específicos}
\section{Alcances Y Limitaciones}
	\subsection{Alcances}
	\subsection{Limitaciones}
\section{Antecendetes Generales de la Empresa}
	\subsection{Identificación de la Empresa}
	\subsection{Misión}
	\subsection{Visión}
	\subsection{Valores}
	\subsection{Mix de Productos}
	\subsection{Procesos Internos}
\section{Analisis de Proceso de gestion financiera}
	\subsection{Identificación de Inputs}
	\subsection{Descripción Actividades}
	\subsection{Descripción de Outputs y sus Características}
	\subsection{Caracterización del cliente}
	\subsection{Análisis de control de calidad}
	\subsubsection{Auditorías constantes Internas}
	\subsection{Identificación y caracterización de las TIC de apoyo}
	
\section{Propuesta de PA de Mejora}
\subsection{Actividades, tiempos, responsables, recursos y Control}
\section{Conclusión}
\section{Bibliografia}

	

