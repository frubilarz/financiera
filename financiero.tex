% Template:     Informe/Reporte LaTeX
% Documento:    Archivo de ejemplo
% Versión:      5.5.4 (16/09/2018)
% Codificación: UTF-8
%
% Autor: Pablo Pizarro R. @ppizarror
%        Facultad de Ciencias Físicas y Matemáticas
%        Universidad de Chile
%        pablo.pizarro@ing.uchile.cl, ppizarror.com
%
% Manual template: [http://latex.ppizarror.com/Template-Informe/]
% Licencia MIT:    [https://opensource.org/licenses/MIT/]

\section{Introducción}
\noindent{Día a día el parque automotriz aumenta y es necesario mantener y tener nuevas señaléticas. Debido a esta necesidad es que nace Serviplott, una empresa la cual se encarga de vender servicios a distintos municipios los cuales requieren nuevas señaléticas y mantención de las señales viales ya existentes para así mejorar la experiencia de conducción.}\vspace{5mm}

\noindent{Toda municipalidad tiene como tarea manejar las señales viales existentes en su sector. Al tener un número alto de señales en los distintos sectores es que las municipalidades optan por externalizar este servicio y se lo dejan a empresas importantes tales como Serviplott.}\vspace{5mm}

\noindent{En el presente trabajo se desarrollará un completo análisis a la gestión de los procesos Industriales de Serviplott, identificando todos los problemas asociados a esta área, para luego proponer un plan de mejora en el ámbito de la Gestión de Procesos.}\vspace{5mm}

\insertimage[]{serviplott/logo2}{width=6cm}{Logo Empresa.}\newpage

\section{Objetivos}
	\subsection{Objetivo General}
	\noindent{El objetivo general consiste en investigar el proceso de gestión financiera llevado a cabo por la empresa “Serviplott Servicios Gráficos y Señalización Vial Ltda” además de analizar como esta apoya sus procesos con el uso de las TI (Tecnologías de la información).}
	\subsection{Objetivos Específicos}
	\begin{itemize}
	\item Analizar los datos obtenidos de la organización, con la finalidad de elaborar una opinión crítica y atingente sobre los diferentes aspectos que conforman el proceso de gestión financiera de Serviplott.
	\item Analizar los datos obtenidos de la organización, con la finalidad de elaborar un plan de acción de mejoras.
	\end{itemize}
	
\section{Antecendetes Generales de la Empresa}

	\subsection{Identificación de la Empresa}
	\noindent{Fundada en 2002, Serviplott es una empresa conformada por profesionales del área de la ingeniería mecánica y la ingeniería en transporte. Estas características le dan a la empresa flexibilidad, rapidez y soporte técnico que demanda el mercado de la infraestructura vial. En Serviplott creen en una relación personalizada con sus clientes y por ello ofrecen tiempos de respuesta de acuerdo a sus particulares necesidades con altos estándares de calidad.}\vspace{5mm}

	\noindent{Inicialmente se encargaban de generar gráficos publicitarias y seguridad, también entregando proyectos de ingeniería en tránsito. Con el tiempo serviplott comenzó a especializarse solamente en soluciones de ingeniería en tránsito, donde se especializó en señales viales.}\vspace{5mm}

	\noindent{A partir de ello, serviplott comienza a licitar en las distintas comunas para así ser una empresa de renombre en mantención e instalación de señaléticas viales. Hoy en día serviplott se encuentra encargado de comunas tales como Santiago, Lo Barnechea, Macul, Ñuñoa.}\vspace{5mm}

	\noindent{Serviplott como empresa se encuentra en una constante actualización de técnicas y tecnologías capacitando a sus distintos trabajadores  en las distintas áreas.}

	\subsection{Misión}
	\noindent{Serviplott es una empresa conformada por profesionales del área de la ingeniería mecánica y la Ingeniería en transportes. Estas características le dan a la empresa la flexibilidad, rapidez y soporte técnico que demanda el mercado de la infraestructura vial. En Serviplott creemos en una relación personalizada con nuestros clientes y por ello ofrecemos tiempos de respuesta de acuerdo a sus particulares necesidades con altos estándares de calidad.}
	
	\subsection{Visión}
	\noindent{Como el mejor proveedor de productos y servicios viales, siendo reconocidos por el mercado como una empresa seria, profesionales y confiable orientada a la satisfacción de sus clientes y de los usuarios del servicio.}
	
	\subsection{Valores}
	\noindent{Nos importa el bienestar de nuestro entorno y colaboradores, preocupándonos de mantener niveles óptimos de seguridad y cumpliendo con las exigencias de las industrias que atendemos. Los materiales y productos utilizados en nuestras obras provienen de proveedores con trayectoria a nivel mundial. Por otro lado nos aseguramos de entregar un servicio de mantención y post venta con respuestas ágiles y eficaces a nuestros clientes.}\vspace{5mm}
	
	\noindent{Es nuestra responsabilidad cumplir con esta promesa y el mayor reconocimiento está en la decisión de nuestros clientes de preferir contratar nuestros servicios. Somos más que energía térmica: estamos comprometidos con el desarrollo y trabajo bien hecho.}
	
	\subsection{Organigrama}
	\noindent{La siguiente figura es una representación gráfica de la empresa, en el cual se muestran las relaciones de jerarquías entre sus diferentes partes.}\vspace{5mm}
	
	\noindent{Este organigrama fue entregado por la empresa que se está analizando, y es un modeleo abstracto y sistemático que permite obtener una idea uniforme y sintética de la estructura formal de la empresa Serviplott.}
	\insertimage[]{serviplott/organigrama}{width=11.2cm}{Organigrama Empresa.}
	
	\subsubsection{Descripción de cargos}
	
	\begin{enumerate}
	\item CEO: Es el director general de la empresa sobre el cual caen todas las responsabilidades.
	\item Gerente General : Es el encargado de coordinar los elementos Finanzas, Producción, Logística y Comercial.
	\item Finanzas
	  \begin{enumerate}
	  \item Contabilidad : Esta área es encarga de emitir facturas y llevar registro de todos los procesos contables de la empresa.
	  \item Cobranza : Existen cobradores encargados de ir a buscar los pagos y de llamar a clientes con deuda vigente.
	  \end{enumerate}
	\item Producción
	  \begin{enumerate}
	    \item Técnicos de producción : Empleados encargados de producir los pedidos.
	    \item Técnicos en terreno : Personal que acude a terreno para montar el pedido.
	  \end{enumerate}
	    \item Logística
	  \begin{enumerate}
	    \item Bodegueros : Encargados de llevar registro del inventario asi como entradas y salidas de la bodega. También genera una solicitud de materias primas cuando están en un bajo nivel.
	    \item Transportistas : Mueven los insumos desde los proveedores hasta la bodega y movilizan a los técnicos en terreno.
	  \end{enumerate}
	\item Comercial
	  \begin{enumerate}
	    \item Ventas : Personal encargado de gestionar las ventas que se hacen.
	    \item Marketing : Área encarga del marketing para promocionar la empresa por distintos medios.\newpage
	  \end{enumerate}
	\end{enumerate}
	
	\subsection{Mix de Productos}
	\noindent{Serviplott Servicios Gráficos y Señalización Vial Ltda es una empresa que ofrece:}
	\begin{itemize}
	\item Mantención de Señales viales.
	\item Instalación de Postes.
	\item Instalación de Señal.
	\insertimage[]{serviplott/mixproducto1}{width=6.5cm}{Señalética.}
	\item Vallas peatonales.
	\insertimage[]{serviplott/mixproducto2}{width=6.5cm}{Valla peatonal.}
	\item Demarcación de Reservados. 
	\item Conos de PVC.
	\insertimage[]{serviplott/mixproducto3}{width=4.5cm}{Cono PVC.}
	\end{itemize}
	
	\vspace{5mm}
	\noindent{\textbf{Mantención de Señales Viales:}}\vspace{3mm} 
	\noindent{Este mix consta de 10 partes que se pueden realizar en una señal ya existente, la cual puede ir desde:}\vspace{3mm}
	\begin{itemize}
	\item Enderezar o aplomar poste. 
	\item Instalación de placa Señal. 
	\item Pintado de poste.
	\item Limpieza de señal.
	\item Cambio de Poste.
	\item Retiro solo placa.
	\item Anclaje de poste.
	\item Traslado de Señal completa.
	\item Reinstalación de Señal completa.
	\item Retiro de Señal completa (cabe destacar esta última debido a que en algunas comunas y en algunas señales tiene cobro). 
	\end{itemize}
	
	\vspace{5mm}
	\noindent{\textbf{Instalación de poste: }}\vspace{3mm}
	\begin{itemize}
	\item Poste Omega de 3.0 con base
	\item Poste Omega con extensión
	\item Poste Omega 2.0
	\item Poste Omega
	\item Poste Omega 3.0
	\item Poste Omega 3.5
	\item Poste 50x50 con base
	\item Poste 50x50
	\end{itemize}
	
	\newpage
	\noindent{\textbf{Instalación de Señales:}}\vspace{3mm}
	\noindent{Para las señales existen distintos tipos de dimensiones y en algunas ocasiones figuras, tal como muestra la siguiente tabla:}\vspace{3mm}
	\insertimage[]{serviplott/dimensiones}{width=15cm}{Dimensiones de Señales.}\vspace{3mm}
	\noindent{\textbf{Vallas peatonales: }\vspace{3mm}
	\noindent{Dentro de este producto se divide en 4 trabajos que se pueden realizar sobre las vallas peatonales:}\vspace{3mm}
	\begin{itemize}
	\item Pintura de vallas peatonales.
	\item Reparación de vallas peatonales.
	\item Instalación de vallas peatonales.
	\item Provisión e instalación de vallas peatonales.
	\end{itemize}
	
	\vspace{3mm}
	\noindent{\textbf{Demarcación de Reservados: }}
	\noindent{La demarcación viene a ser la pintura que se coloca en los distintos espacios para reservados que sean solicitados ya sea por espacios públicos como demarcación de estacionamientos para personas con capacidad reducida, o espacios solicitado de privados a la municipalidades como puede ser el espacio de taxis y colectivos.}
	
	\newpage
	\subsection{Procesos internos}
	\noindent{La siguiente figura (Procedimiento de ejecucion de proyectos) es un cuadro de los procedimientos que se hacen dentro de la empresa para la ejecución de proyectos.}
	
	\insertimage[]{serviplott/procedimientoejecuciondeproyectos}{width=16cm}{Procedimiento de ejecucion de proyectos}
	\noindent{Esta figura referencia todos los procesos de la empresa serviplott en tanto a logística}
	\insertimage[]{serviplott/procesologistica}{width=14cm}{Procesos de logística}
	
\section{Analisis de Proceso de gestion financiera}

	\subsection{Identificación de Inputs}
	
	\noindent{Los inputs para el proceso financiero de Serviplott son los que se muestran a continuación:}
	
	\subsubsection{Inputs}
	
	\noindent{Materiales, equipamiento, información, recursos financieros,recursos humanos o condiciones medio ambientales requeridas para llevar a cabo el proceso financiero.}\vspace{3mm}
	
	\noindent{Proveedores: Personas (clientes) que abastecen al proceso con sus inputs.Estos pueden ser:}
	
	\begin{itemize}
	\item Depósitos bancarios.
	\item Cheques.
	\item Transferencias electrónicas.
	\item Vales vista.
	\item Otros.(Aunque no me gusta la idea de colocar etc u otros, pero en el otro informe aparecía).
	\end{itemize}
	
	\noindent{Y sus facturas correspondientes son elevadas al término de las fases del pago de pie por un proyecto y la finalización de este.}\\\\
	\noindent{Otros de los inputs que se logran identificar y que son altamente involucrados en la gestión financiera, son los siguientes:}.
	
	\begin{itemize}
	\item Boleta de servicios básicos. 
	\item Facturación.
	\item Formulario de Impuestos.
	\item Boleta de Honorarios. 
	\item Boleta de Pago.
	\item Boleta de honorarios de diseño. 
	\item Boleta de pago de dominio.
	\item Boleta de pago de hosting.
	\item Boleta de pago de artículos de oficina.
	\item Entre otros.(Aunque no me gusta la idea de colocar etc u otros, pero en el otro informe aparecía).
	\end{itemize}
	
	\subsection{Descripción Actividades}
	
	\subsubsection{Proceso de gestión financiera}
	
	\noindent{En cada una de las etapas del proceso de gestión financiera se realizan las siguientes actividades:}
	
	\begin{itemize}
	\item Recopilacion de Informacion: En esta etapa se recopila toda información necesaria o útil para las actividades del negocio, tales como, hechos económicos, información de mercado, hechos sociales, hechos ambientales, entre otros.
	\item Análisis de Requisitos​: Etapa en donde se procede a clasificar y analizar la información, de acuerdo a las políticas propias de la empresa, registrando dicha información debidamente. Además se pretende definir los requerimientos futuros de la empresa para su funcionamiento tomando en cuenta los objetivos a los que apunta.
	\item Definición de presupuesto​: Etapa en la cual se busca definir el presupuesto disponible para las actividades de la empresa y su correspondiente asignación en base a la información analizada. Además se fijan las políticas de la empresa sobre el manejo y uso de los presupuestos estimados.
	\item Contabilidad​: Etapa que busca definir los costos de las operaciones del periodo actual, además de identificar los elementos de costo que afectan las actividades de la empresa.
	\item Fijación de precios​: Etapa en la que se establecen las tarifas por productos y servicios prestados a los clientes. Para el caso de tratarse de un servicio especial el precio de este está fuertemente ligado al valor de las monedas.
	\item Control y seguimiento: Etapa que busca controlar las reglas de operación de los programas definidos y mantener un seguimiento a través de informes de resultados y el monitoreo de indicadores, planteando planes de monitoreo para validar los indicadores e informes que se han presentado.
	\item Resultados​: Etapa en donde se publican los resultados e informes de las actividades desarrolladas por la empresa.
	\end{itemize}
	
	\noindent{\textbf{Indicadores}}
	\\\\
	\noindent{\textbf{Indicadores de liquidez}}
	\vspace{3mm}
	
	\noindent{La liquidez de una organización es juzgada por la capacidad para saldar las obligaciones a corto plazo que se han adquirido a medida que éstas se vencen. Se refieren no solamente a las finanzas totales de la empresa, sino a su habilidad para convertir en efectivo determinados activos y pasivos corrientes.}
	
	\insertimage[]{serviplott/indicadoresliquidez}{width=6.5cm}{Indicador de liquidez}
	
	\noindent{\textbf{Indicadores de eficiencia}}
	\vspace{3mm}
	
	\noindent{Establecen la relación entre los costos de los insumos y los productos de proceso; determinan la productividad con la cual se administran los recursos, para la obtención de los resultados del proceso y el cumplimiento de los objetivos. Los indicadores de eficiencia miden el nivel de ejecución del proceso, se concentran en el Cómo se hicieron las cosas y miden el rendimiento de los recursos utilizados por un proceso. Tienen que ver con la productividad.}
	
	\insertimage[]{serviplott/indicadoreseficiencia}{width=6.5cm}{Indicador de eficiencia}
	
	\noindent{\textbf{Indicadores de desempeño}}
	\vspace{3mm}
	
	\noindent{Es un instrumento de medición de las principales variables asociadas al cumplimiento de los objetivos y que a su vez constituyen una expresión cuantitativa y/o cualitativa de lo que se pretende alcanzar con un objetivo específico establecido.}
	
	\insertimage[]{serviplott/indicadoresdesempeno}{width=7cm}{Indicador de desempeño}
	
	\noindent{\textbf{Indicadores de productividad}}
	\vspace{3mm}
	
	\noindent{La productividad está asociada a la mayor producción por cada hombre dentro de la empresa y al manejo razonable de la eficiencia y la eficacia.}
	
	\insertimage[]{serviplott/indicadoresproductividad}{width=12cm}{Indicador de productividad}
	
	\noindent{\textbf{Indicadores de endeudamiento}}
	\vspace{3mm}
	
	\noindent{Tienen por objeto medir en qué grado y de qué forma participan los acreedores dentro del financiamiento de la empresa. De la misma manera se trata de establecer el riesgo que incurren tales acreedores, el riesgo de los dueños y la conveniencia o inconveniencia de un determinado nivel de endeudamiento para la empresa.}
	
	\insertimage[]{serviplott/indicadoresendeudamiento}{width=6.5cm}{Indicador de endeudamiento}
	
	\noindent{\textbf{Indicadores de diagnóstico financiero}}
	\vspace{3mm}
	
	\noindent{El diagnóstico financiero es un conjunto de indicadores qué, a diferencia de los indicadores de análisis financiero, se construyen no solamente a partir de las cuentas del Balance General sino además de cuentas del Estado de Resultados, Flujo de Caja y de otras fuentes externos de valoración de mercado. Esto conlleva a que sus conclusiones y análisis midan en términos más dinámicos, y no estáticos, el comportamiento de una organización en términos de rentabilidad y efectividad en el uso de sus recursos.}
	
	\insertimage[]{serviplott/indicadoresf1}{width=8cm}{Indicador financiero}
	\insertimage[]{serviplott/indicadoresf2}{width=8cm}{Indicador financiero}
	
	\noindent{\underline{Humanware}}
	\vspace{3mm}
	
	\begin{itemize}
	\item Gerencia: Persona o grupo de personas encargadas de la toma de decisiones de la empresa, además de ser quienes definen las políticas que dirigen la misma.
	\item Contador General​: Persona encargada de llevar a cabo la contabilidad de la empresa y que entre sus funciones está, llevar el control sobre las distintas actividades que constituyen el movimiento contable y que dan lugar a los balances y demás reportes financieros.
	\item Contador-Auditor​ : El contador auditor es aquella persona que verifica y certifica la información generada por el contador general, esto en busca de posibles errores y “fugas” que se puedan estar generando y no estén siendo incluidas en los informes del contador general. Este Contador, al tener mayor responsabilidad y peso, también tiene un nivel de jerarquía alto, pudiendo dirigirse directamente a cualquier empleado y sistema de información de la empresa para validar los datos.
	\item Auditor Externo​ : Este Auditor (que generalmente es un contador-auditor en el caso financiero) es el encargado de verificar y certificar que la información contenida en los informes de gestión financiera sean los correctos y se adapten a los requeridos por las normas vigentes. Generalmente este auditor es contratado para una auditoría a final de año dirigida al director, socios y accionistas, aunque también se le puede solicitar en casos específicos de pérdidas de dinero por robo, fraude o negligencia. 
	\end{itemize}
	
	\noindent{\underline{Software}}
	\vspace{3mm}
	
	\noindent{Los principales elementos de software utilizados para apoyar el proceso son: }
	
	\begin{itemize}
	\item MS-Office: Paquete de softwares de oficina, contiene el popular programa de ofimática “Microsoft Excel”, ampliamente utilizado por contadores para presentacion de informacion en forma de Tabla.
	\item SAP: Serviplott procesa toda su información financiera, productiva y comercial a través de SAP. Este ERP mantiene un registro y control de acceso a información a nivel de consulta y proceso, mediante la asignación de perfiles de usuarios.
	\item SQL: Sistema de Base de Datos utilizado para contener información de todos los ámbitos de la empresa, entre ellos, financieros.
	\end{itemize}
	
	\noindent{\underline{Hardware}}
	\vspace{3mm}
	
	\noindent{Los principales hardware utilizados para apoyar el proceso son:}
	
	\begin{itemize}
	\item Computadoras personales (Notebooks y PC): Estos son provistos por la empresa a sus empleados y son ampliamente utilizados para el cálculo, el almacenamiento y la comunicación que sirve para los procesos financieros.
	\item Servidores: Estos están alojados dentro y fuera de la empresa y su contenido es manejado por el administrador de servicios TI, contienen información general y específica de la empresa, entre ella información de ventas y compras, que apoyan el proceso de gestión financiera.
	\end{itemize}
	
	\subsection{Descripción de Outputs y sus Características}
	
	\subsubsection{Output}
	
	\noindent{El producto o servicio creado por el proceso; el que se entrega al cliente. En este caso, son los servicios que presta la empresa, como también el estado actual de cada proceso, como son los estados financieros, estadísticas de la empresa, memorias, entre otras cosas con relación a la empresa.}\\\\
	\noindent{Cliente: Los que utilizan su output. Tanto si sus clientes son internos como externos, utilizan el output como un input para sus procesos de trabajo.}\\\\
	\noindent{Los outputs que el proceso financiero suelen ser presentados en formato de informe, los más destacados de estos son:}
	
	\subsubsection{Memorias anuales}
	
	\noindent{Toda la información correspondiente a los estados financieros del periodo anual completo, con fecha del último día de cada año, con información general y específica de toda la institución.}
	
	\subsubsection{Estados financieros}
	
	\noindent{Toda la información correspondiente a los estados financieros del periodo, separados por cada tres meses, tales como:}
	
	\begin{itemize}
	\item Balance General: Informe financiero contable que refleja la situación económica y financiera de la empresa.
	\item Estado Resultado: Estado financiero que muestra ordenada y detalladamente la forma de cómo se obtuvo el resultado.
	\item Análisis Razonado: Informe complementario a los estados financieros y que debe ser leído junto con los Estados Financieros Consolidados.
	\end{itemize}
	
	\subsubsection{Análisis razonados}
	
	\noindent{Informe por períodos de tres meses, el cual plantea el análisis de sus respectivos estados financieros. Estos informes dependen mucho del cliente, de ser una municipalidad por ley de transparencia deben estar disponibles para toda la ciudadanía.}
	
	\subsubsection{Informes Auditoría Externa}
	
	\noindent{Corresponden a los informes certificados de auditores externos a la empresa, los cuales verifican la veracidad y la calidad de los procedimientos con respecto a las normas correspondientes.}\\
	
	\noindent{Los informes y reportes anteriores, no son más que el conjunto de otros outputs, como los destacados a continuación:}\\
	
	\noindent{\textbf{Presupuesto}}
	\vspace{3mm}
	
	\noindent{Información sobre los ingresos y gastos previstos para el periodo determinado en el proceso.}\\
	
	\noindent{\textbf{Reportes}}
	\vspace{3mm}
	
	\noindent{Reportes sobre distintos aspectos considerados en la empresa, como por ejemplo:}
	
	\begin{itemize}
	\item Reportes de gastos.
	\item Reportes de presupuestos.
	\item Reportes de costo/beneficio.
	\item Reportes de flujo de dinero
	\item Entre otros.
	\end{itemize}
	
	\noindent{\textbf{Ratios}}
	\vspace{3mm}
	
	\noindent{Información de los ratios del periodo, de tal forma de conocer el estado de cumplimiento de estos.}
	
	\begin{itemize}
	
	\item Liquidez corriente:
	  \begin{itemize}
	   \item Activos corrientes / Pasivos corrientes
	  \end{itemize}
	 \item Razón ácida:
	  \begin{itemize}
	   \item (Activos corrientes-existencias-gastos anticipados)/Pasivos corrientes
	  \end{itemize}  
	 \item Razón de endeudamiento:
	  \begin{itemize}
	   \item (Pasivos corrientes + pasivos no corrientes)/Patrimonio total
	  \end{itemize}
	 \item Proporción deuda corto plazo
	  \begin{itemize}
	   \item Total deuda corriente / Total deuda
	  \end{itemize}
	 \item Proporción deuda largo plazo
	  \begin{itemize}
	   \item Total deuda no corriente / Total deuda
	  \end{itemize}
	 \item Cobertura gastos financieros
	  \begin{itemize}
	   \item Resultado antes de impuesto+Costos financieros/Costos financieros
	  \end{itemize}
	 \item Rentabilidad del patrimonio
	  \begin{itemize}
	   \item Resultado neto del período / Patrimonio promedio
	  \end{itemize}
	 \item Rentabilidad del activo
	  \begin{itemize}
	   \item Resultado neto del período / Total activos promedio
	  \end{itemize}
	 \item Rendimiento de activos operacionales
	  \begin{itemize}
	   \item Resultado operacional del período / Total activos operacionales promedio
	  \end{itemize}
	 \item Utilidad por acción
	  \begin{itemize}
	   \item Resultado neto al cierre / Cantidad acciones suscritas y pagadas
	  \end{itemize}
	 \item Ebitda
	  \begin{itemize}
	   \item Ingresos de explotación - Costo de ventas (sin Depreciación y amortización) - Costos de distribución - Otros gastos, por función + Depreciación y amortización
	  \end{itemize}
	 \item Margen Ebitda
	  \begin{itemize}
	   \item Ebitda / Ingresos de explotación
	  \end{itemize}
	 \end{itemize}
	 
	\noindent{\textbf{Índices}}
	\vspace{3mm}
	
	\noindent{Información que se extrae luego del análisis de los ratios del periodo.}
	
	\begin{itemize}
	\item Índice de Liquidez: Se extrae al analizar la información entregada por los ratios.
	  \begin{itemize}
	  \item Liquidez corriente
	  \item Test ácido
	  \item Capital de Trabajo
	  \item Plazo promedio de cuenta por cobrar
	  \item Índice rotación de cuentas por cobrar
	  \item Plazo promedio de cuentas por pagar
	  \item Rotación de inventario
	  \end{itemize}
	\end{itemize}
	
	\begin{itemize}
	\item Índice de Utilidad: Se extrae al analizar la información entregada por los ratios.
	  \begin{itemize}
	  \item Margen de Utilidad Bruta
	  \item Margen de Utilidad Neta
	  \item Rentabilidad sobre activos
	  \item Rotación de Activos
	  \item Du-Pont
	  \item Rentabilidad sobre el patrimonio
	  \end{itemize}
	\end{itemize}
	
	\begin{itemize}
	\item Índice de Endeudamiento: Se extrae al analizar la información entregada por los ratios:
	  \begin{itemize}
	  \item Índice de endeudamiento sobre activo
	  \item Leverage
	  \item Deuda de largo plazo sobre el patrimonio
	  \end{itemize}
	\end{itemize}
	 
	\subsection{Caracterización del cliente}
	
	\noindent{En este punto se dividen dos tipos de clientes los cuales son internos y externos.}
	
	\subsubsection{Clientes internos}
	
	\noindent{Entre los los clientes internos de los procesos financieros de Serviplott, se encuentran:}\\
	
	\noindent{\underline{Propietarios y gerentes:} En general, la parte directiva de la empresa necesita conocer los estados financieros de esta, para así, poder tomar decisiones importantes sobre el negocio y que afectarán la continuidad de las operaciones.}\\
	
	\noindent{\underline{Empleados:} Los empleados también requieren de la información financiera para desempeñar algunas actividades en situaciones específicas. Por ejemplo, para acuerdos de negociación colectiva en negociaciones en torno a las remuneraciones. También sirve es necesario que tengan en cuenta ciertos índices para dirigir sus esfuerzos a ciertas actividades con más prioridad que otras.}\\
	
	\noindent{Otros clientes pueden ser definidos como:}
	
	\begin{itemize}
	\item Clientes de servicios básicos.
	\item Clientes de contabilidad.
	\item Clientes de sala de producción, cliente de servicio de bodega. 
	\item Clientes de servicio de envío.
	\item Área de RR.HH.
	\item Área de Administración de Decisiones.
	\end{itemize}
	
	\subsubsection{Clientes externos}
	
	\noindent{\underline{Municipalidades:} Serviplott tiene como principal clientes distintas municipales tales como Santiago, Macul, Ñuñoa, entre otra. Las Municipalidades tienden a ser licitaciones por un periodo no inferior de 4 años por licitación en la cual cada comuna acepta la postulación de la empresa y los costos asociados a la misma. Las municipalidades de santiago y ñuñoa tienen distintos formas de comprar productos, se muestran las ofertas económicas de ambas comunas.}\\
	
	\noindent{\underline{Empresas:} Existen empresas que solicitan impresiones para colocar en sus instalaciones tales como demarcaciones de zonas de riesgo, ingreso solo de personal autorizado, zonas de silencio, zonas limpias, etc.}\\
	
	\insertimage[]{serviplott/stgo}{width=9cm}{Oferta económica, comuna de Santiago}
	\insertimage[]{serviplott/nunoa1}{width=9cm}{{Oferta económica, comuna de Ñuñoa 1/3}}
	\insertimage[]{serviplott/nunoa2}{width=9cm}{{Oferta económica, comuna de Ñuñoa 2/3}}
	\insertimage[]{serviplott/nunoa3}{width=9cm}{Oferta económica, comuna de Ñuñoa 3/3}
	
	\subsection{Análisis de control de calidad e identificación de sistemas de control}
	
	\noindent{La empresa serviplott, tiene sus sitemas de control de calidad, los cuales fueron mencionados anteriormente, juntos con sus riesgos y comités relacionados. Al mismo tiempo, esta cuenta con auditorias tanto internas como externas.}
	
	\subsubsection{Auditoría externa}
	
	\noindent{Al funcionar como un inversor, se le entrega (como métrica de calidad financiera) el flujo de caja y estados de resultados, haciendo especial énfasis en los indicadores de rendimiento y rentabilidad (ROI y el ROE).\\\\ Servicio de impuestos internos: Al servicio de impuestos internos(SII), se le debe entregar toda la información que se pertinente en cuanto a los tributos financieros que deba declarar la empresa.\\\\ SVS: La Superintendencia de Valores y Seguros (SVS) tiene entre sus objetivos principales velar por la transparencia de los mercados que supervisa, mediante la oportuna y amplia difusión de la información pública que mantiene y, colaborar en el conocimiento y educación de inversionistas, asegurados y público en general. Todos ellos, elementos esenciales para el desarrollo y correcto funcionamiento de dichos mercados.}
	
	\subsubsection{Auditoría interna}
	
	\noindent{Este tipo de auditorías no están definidas en un tiempo específico, sino que están orientadas a “sorprender” al personal en errores o negligencias que puedan estar cometiendo, si bien generalmente una Auditoría Interna es tomada con cierta desconfianza por posibles conflictos de interés, en el caso de Serviplott. suelen traer información verídica, esto es porque los auditores no tienen generalmente roce con el personal, por ende no hay un ambiente de alta tolerancia a errores, además de eso, en caso de que las auditorías internas no concuerden con la externa, es posible que se realice un sumario y una investigación sobre mala praxis de parte del auditor, penalizando a este o siendo desvinculado de la empresa, Serviplott al ser una empresa muy grande necesita de estas auditorías para mantener a sus empleados en un estado de alerta. Este tipo de auditorías tienen un output de informe bastante más pequeño que el anual, generalmente es realizado por el contador auditor o en su defecto por un contador general en búsqueda de un error o dato extraño en específico. Sin embargo, estos documentos, al contar con información muy detallada en ciertos casos, no están abiertos al público y suelen ser orientadas a una gerencia en específico y no a la Gerencia General, además de eso, generalmente son aplicados dentro de una misma instalación, por ende, extraña vez llega a Gerencia General o a otra Gerencia de alto nivel Jerárquico, sólo en casos específicos los que puedan afectar la estabilidad de la empresa se llega a ese extremo (Robo, fraude, negligencia).}
	
	\subsubsection{Ratios}
	
	\noindent{Alguno de los ratios que está ocupando Serviplott son los siguientes:}
	
	\insertimage[]{serviplott/Calidad1}{width=9cm}{Tabla con Indicadores de cumplimiento de Índices Legales y Normativos (valores actualizados a marzo del 2018)}
	
	\subsubsection{Índice de liquidez general para Serviplott (Corredora de Bolsa):}
	
	\noindent{El ratio de liquidez general se obtiene dividiendo el activo corriente entre el pasivo corriente.}
	
	\insertimage[]{serviplott/Calidad2}{width=9cm}{Índice de liquidez general}
	
	\subsubsection{Razón de endeudamiento (Corredora de Bolsa):}
	
	\noindent{Este ratio o índice evalúa la relación entre los recursos totales a corto plazo y largo plazo aportados por los acreedores y los aportados por los propietarios de la empresa.}
	
	\insertimage[]{serviplott/Calidad4}{width=11cm}{Razón de endeudamiento}
	
	\subsubsection{Razón Patrimonial para Serviplott (Corredora de Bolsa):}
	
	\noindent{Este indicador mide el grado de compromiso del patrimonio para con los acreedores de la empresa. No debe entenderse como que los pasivos se puedan pagar con patrimonio, puesto que, en el fondo, ambos constituyen un compromiso para la empresa.}
	
	\insertimage[]{serviplott/Calidad5}{width=11cm}{Razón patrimonial}
	
	\subsection{Identificación y caracterización de las TIC de apoyo}
	
	\subsubsection{Sistema de correos electrónicos}
	
	\noindent{Este sistema es uno de los principales medios de comunicación que tiene serviplott para interactuar con clientes ya sean nuevos o ya existentes, como también con sus colaboradores.  Las reuniones son solicitadas y pactadas por este medio donde se deja en claro el lugar hora y fecha y es enviado con copia los involucrados.}\\
	
	\noindent{En este caso el proveedor de correos es el servidor hosting y para un manejo de llegada y envio utilizan la herramienta outlook desktop.}
	
	\subsubsection{Sistema de documentación}
	
	\noindent{La empresa genera todo tipo de documentación con la herramienta ofimáticas “Microsoft Word”. A pesar de que su sistema de almacenamiento en la nube también dispone de herramientas de ofimática, los empleados realizan toda la documentación en Microsoft Word.}\\
	
	\noindent{La documentación que se genera en esta herramienta son las minutas, hojas de correcciones, requerimientos funcionales, entre otros. También cabe destacar que los requerimientos que obtienen de los empleados los recopilan mediante la técnica de “historias de usuario”, apuntandolos en papel en primer lugar, y luego traspasando a digital mediante Microsoft Word. Empresa Proveedora: Microsoft (Microsoft Word)}
	
	\subsubsection{Sistema de Administración financiero y contable}
	
	\noindent{Para llevar la contabilidad o los presupuestos de los proyectos, los empleados y encargados ocupan las plantillas de Microsoft Excel. Todas las cotizaciones de los proyectos son documentadas en estas planillas, teniendo todo un detalle en los hitos y actividades según su costo monetario. La empresa no maneja herramientas de proyección financiera, sólo se dedica a registrar sus movimientos. Empresa Proveedora: Microsoft (Microsoft Excel)}
	
	\subsubsection{Sistema de planificación}
	
	\noindent{Cabe destacar que los requerimientos que obtienen de los empleados los recopilan mediante la técnica de “historias de usuario”, apuntandolos en papel en primer lugar, y luego traspasando a digital mediante Microsoft Word. En esos momentos hacen planificaciones también en formato papel, para luego traspasarlo al sistema de correo electrónico. Empresa Proveedora: Microsoft (Microsoft Word))}
	
	\subsubsection{Sistema de facturación}
	
	\noindent{Para facturar los proyectos que realiza la empresa, Serviplott usa el sistema del Servicio de Impuestos Internos de facturación electrónica. No posee facturas en papel. Empresa Proveedora: S.I.I (Servicio de Impuestos Internos).}
	
	\subsubsection{Sistema Santiago.serviplott}
	
	\noindent{Santiago por licitación solicita a serviplott un sistema para llevar los cobros y actividades de cada mes. este sistema es hecho a medida por Fernando Rubilar Zepeda, quien desarrolla dos aplicaciones una web y otra android para los trabajadores en terreno.}
	
	\insertimage[]{serviplott/sistema}{width=14cm}{Sistema Santiago.serviplott}
	
	\subsubsection{Sistema de almacenamiento en la nube}
	
	\noindent{Los sistemas de almacenamiento en la nube les permiten, a los empleados de la empresa, depositar y gestionar los datos necesarios para trabajar en los proyectos, además de guardar los requerimientos de los clientes. Actualmente la empresa está suscrita a cuentas de Google para ocupar su sistema de la nube llamado “Google Drive”. A pesar de que “Google Drive” proporciona herramientas de ofimática, solo es ocupado como sistema de almacenamiento. Empresa proveedora: Google.}
	
	\subsubsection{Sistema de servidores remotos}
	
	\noindent{A diferencia del sistema de almacenamiento en la nube, los sistemas de servidores remotos se encargan de almacenar todos los recursos que necesita el proyecto y para servir a los procesos necesarios que estén encargados de hacer funcionar el software. Estos sistemas son manejados por paneles web. También utilizan conexiones con el protocolo de ftp o ftps para la transferencia de archivos con interfaces gráficas. Todos los servidores que administra la empresa están configurados para seguir el modelo cliente-servidor. Todos estos servidores son provistos por Hostings, debido a que la empresa no tiene uno propio. Empresa proveedora: Microsoft.}
	
\section{Propuesta de PA de Mejora}

	\subsection{Actividades, tiempos, responsables, recursos y Control}

	\noindent{Dentro de las opciones de mejoras que se pueden plantear, encontramos que solicitar soportes de sistemas a medida es una alternativa para entregar mayor integración de la información de las diferentes áreas, ya que dentro de sus principales problemas actualmente está la dispersión de las diferentes fuentes de información. La mejora provocaría poder acceder a datos en tiempos más óptimos no buscando papeles y/o informes.}\\
	
	\noindent{Es importante destacar, que serviplott, tiene muy buen registro de lo que solicita y se hace en la comuna de santiago, modelo el cual puede ser replicado en distintas comunas y así tener una mejora en el proceso financiero, esto haciendo que una vez cerrado el mes estén los montos y poder levantar la factura.}\\
	
	\noindent{Este modelo pretende juntar los centros de costos y así tener un flujo financiero aún más expedito.}\\
	
	\noindent{Dentro de nuestra propuesta orientamos una metodología en espiral la cual cuenta de 7 pasos la cual cada vez que ocurre un paso se vuelve al inicio para analizar el flujo completo de la empresa.}\\
	
	\begin{enumerate}
	\item Selección de oportunidades de mejora: revisión de antecedentes, listar problemas, jerarquizar los más importantes, escoger y chequear el problema.
	\item Cuantificación y subdivisión: clarificar, subdividir y cuantificar el problema, escoger subdivisión a base de datos.
	\item Análisis de causas raíces: listar causas por subdivisión, agrupar las causas, cuantificar y seleccionar causas.
	\item Nivel de desempeño requerido (metas): definir el nivel del indicador, establecer propuestas.
	\item Diseño y programación de soluciones: listar posibles soluciones, seleccionar las soluciones más factibles y potenciales, programar las actividades de cada solución.
	\item Implantación de soluciones: verificar (reajustar) el cumplimiento del programa, chequear los niveles alcanzados por los indicadores, evaluar el impacto de las mejoras incorporadas.
	\item Establecimiento de acciones de garantía: normalizar prácticas operativas, entrenamiento en los nuevos métodos, incorporar el control del departamento, reconocer y definir resultados.
	\end{enumerate}
	
	\insertimage[]{serviplott/propuesta_1}{width=14cm}{Propuesta de PA de mejora 1/2}
	\insertimage[]{serviplott/propuesta_2}{width=14cm}{Propuesta de PA de mejora 2/2}
	
	\noindent{Explicamos que se trabajará con metodología espiral debido a que cada empresa solicita servicios a los distintos procesos y levantar actividades en el mismo tiempo para todos los clientes no es viable debido a que los clientes habituales, las municipalidades, tienen particularidades por lo que se pretende llegar a una solución única en cuanto al proceso financiero, donde puedan revisar todos los clientes cuanto es lo que corresponde a su flujo financiero del mes en curso y de los distintos meses en el pasado.}\\

\section{Conclusión}

	\noindent{El sistema de gestión financiero de Serviplott, ciertamente es bastante robusto, tiene procesos bien definidos, pero mejorables sobre todo en la entrega de los datos finales.}\\
	
	\noindent{Lo anteriormente descrito, es observable en la diversidad de documentos y estimadores que genera su proceso financiero, ya sean, distintos tipos de reportes, documentos más complejos como las memorias de costos anuales, gran variedad de estados financieros y un amplio espectro de KPIs asociados a las actividades del negocio, que permiten medir de forma concreta y de distintas perspectivas, los logros y oportunidades de mejora que tiene la empresa.}\\
	
	\noindent{La gran diversidad de output generados por el proceso financiero de serviplott, tiene que ver con las condiciones particulares de cada cliente, en el caso del informe las municipalidades. Esto por un lado los ayuda a tener diversidad entre los ingresos y a la vez es un riesgo debido a que cada municipalidad tiene un cobro y trato distinto.}\\
	
	\noindent{Con respecto  al PA, creemos que  la mejor forma de mantener la mejora y calidad del proceso es aplicar pasos de la mejora continua. como mencionamos previamente, la implementación de sistemas a medida, para así tener una integración entre estos y poder tener la información de manera directa y en el momento que uno desee. Por tanto, la mejora continua, daría facilidades para poder identificar esas oportunidades de mejora e implementar un mecanismo de mejora en el proceso cuando sea necesario.}\\

\section{Bibliografia}

	

