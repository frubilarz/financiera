% Template:     Informe/Reporte LaTeX
% Documento:    Archivo de ejemplo
% Versión:      5.5.4 (16/09/2018)
% Codificación: UTF-8
%
% Autor: Pablo Pizarro R. @ppizarror
%        Facultad de Ciencias Físicas y Matemáticas
%        Universidad de Chile
%        pablo.pizarro@ing.uchile.cl, ppizarror.com
%
% Manual template: [http://latex.ppizarror.com/Template-Informe/]
% Licencia MIT:    [https://opensource.org/licenses/MIT/]

\section{Introducción}
\noindent{Día a día el parque automotriz aumenta y es necesario mantener y tener nuevas señaléticas. Debido a esta necesidad es que nace Serviplott, una empresa la cual se encarga de vender servicios a distintos municipios los cuales requieren nuevas señaléticas y mantención de las señales viales ya existentes para así mejorar la experiencia de conducción.}\vspace{5mm}

\noindent{Toda municipalidad tiene como tarea manejar las señales viales existentes en su sector. Al tener un número alto de señales en los distintos sectores es que las municipalidades optan por externalizar este servicio y se lo dejan a empresas importantes tales como Serviplott.}\vspace{5mm}

\noindent{En el presente trabajo se desarrollará un completo análisis a la gestión de los procesos Industriales de Serviplott, identificando todos los problemas asociados a esta área, para luego proponer un plan de mejora en el ámbito de la Gestión de Procesos.}\vspace{5mm}

\insertimage[]{serviplott/logo2}{width=6cm}{Logo Empresa.}\newpage

\section{Objetivos}
	\subsection{Objetivo General}
	\noindent{El objetivo general consiste en investigar el proceso de gestión financiera llevado a cabo por la empresa “Serviplott Servicios Gráficos y Señalización Vial Ltda” además de analizar como esta apoya sus procesos con el uso de las TI (Tecnologías de la información).}
	\subsection{Objetivos Específicos}
	\begin{itemize}
	\item Analizar los datos obtenidos de la organización, con la finalidad de elaborar una opinión crítica y atingente sobre los diferentes aspectos que conforman el proceso de gestión financiera de Serviplott.
	\item Analizar los datos obtenidos de la organización, con la finalidad de elaborar un plan de acción de mejoras.
	\end{itemize}
	
\section{Antecendetes Generales de la Empresa}

	\subsection{Identificación de la Empresa}
	\noindent{Fundada en 2002, Serviplott es una empresa conformada por profesionales del área de la ingeniería mecánica y la ingeniería en transporte. Estas características le dan a la empresa flexibilidad, rapidez y soporte técnico que demanda el mercado de la infraestructura vial. En Serviplott creen en una relación personalizada con sus clientes y por ello ofrecen tiempos de respuesta de acuerdo a sus particulares necesidades con altos estándares de calidad.}\vspace{5mm}

	\noindent{Inicialmente se encargaban de generar gráficos publicitarias y seguridad, también entregando proyectos de ingeniería en tránsito. Con el tiempo serviplott comenzó a especializarse solamente en soluciones de ingeniería en tránsito, donde se especializó en señales viales.}\vspace{5mm}

	\noindent{A partir de ello, serviplott comienza a licitar en las distintas comunas para así ser una empresa de renombre en mantención e instalación de señaléticas viales. Hoy en día serviplott se encuentra encargado de comunas tales como Santiago, Lo Barnechea, Macul, Ñuñoa.}\vspace{5mm}

	\noindent{Serviplott como empresa se encuentra en una constante actualización de técnicas y tecnologías capacitando a sus distintos trabajadores  en las distintas áreas.}

	\subsection{Misión}
	\noindent{Serviplott es una empresa conformada por profesionales del área de la ingeniería mecánica y la Ingeniería en transportes. Estas características le dan a la empresa la flexibilidad, rapidez y soporte técnico que demanda el mercado de la infraestructura vial. En Servilplott creemos en una relación personalizada con nuestros clientes y por ello ofrecemos tiempos de respuesta de acuerdo a sus particulares necesidades con altos estándares de calidad.}
	
	\subsection{Visión}
	\noindent{Como el mejor proveedor de productos y servicios viales, siendo reconocidos por el mercado como una empresa seria, profesionales y confiable orientada a la satisfacción de sus clientes y de los usuarios del servicio.}
	
	\subsection{Valores}
	\noindent{Nos importa el bienestar de nuestro entorno y colaboradores, preocupándonos de mantener niveles óptimos de seguridad y cumpliendo con las exigencias de las industrias que atendemos. Los materiales y productos utilizados en nuestras obras provienen de proveedores con trayectoria a nivel mundial. Por otro lado nos aseguramos de entregar un servicio de mantención y post venta con respuestas ágiles y eficaces a nuestros clientes.}\vspace{5mm}
	
	\noindent{Es nuestra responsabilidad cumplir con esta promesa y el mayor reconocimiento está en la decisión de nuestros clientes de preferir contratar nuestros servicios. Somos más que energía térmica: estamos comprometidos con el desarrollo y trabajo bien hecho.}
	
	\subsection{Organigrama}
	\noindent{La siguiente figura es una representación gráfica de la empresa, en el cual se muestran las relaciones de jerarquías entre sus diferentes partes.}\vspace{5mm}
	
	\noindent{Este organigrama fue entregado por la empresa que se está analizando, y es un modeleo abstracto y sistemático que permite obtener una idea uniforme y sintética de la estructura formal de la empresa Serviplott.}
	\insertimage[]{serviplott/organigrama}{width=11.2cm}{Organigrama Empresa.}
	
	\subsubsection{Descripción de cargos}
	
	\begin{enumerate}
	\item CEO: Es el director general de la empresa sobre el cual caen todas las responsabilidades.
	\item Gerente General : Es el encargado de coordinar los elementos Finanzas, Producción, Logística y Comercial.
	\item Finanzas
	  \begin{enumerate}
	  \item Contabilidad : Esta área es encarga de emitir facturas y llevar registro de todos los procesos contables de la empresa.
	  \item Cobranza : Existen cobradores encargados de ir a buscar los pagos y de llamar a clientes con deuda vigente.
	  \end{enumerate}
	\item Producción
	  \begin{enumerate}
	    \item Técnicos de producción : Empleados encargados de producir los pedidos.
	    \item Técnicos en terreno : Personal que acude a terreno para montar el pedido.
	  \end{enumerate}
	    \item Logística
	  \begin{enumerate}
	    \item Bodegueros : Encargados de llevar registro del inventario asi como entradas y salidas de la bodega. También genera una solicitud de materias primas cuando están en un bajo nivel.
	    \item Transportistas : Mueven los insumos desde los proveedores hasta la bodega y movilizan a los técnicos en terreno.
	  \end{enumerate}
	\item Comercial
	  \begin{enumerate}
	    \item Ventas : Personal encargado de gestionar las ventas que se hacen.
	    \item Marketing : Área encarga del marketing para promocionar la empresa por distintos medios.\newpage
	  \end{enumerate}
	\end{enumerate}
	
	\subsection{Mix de Productos}
	\noindent{Serviplott Servicios Gráficos y Señalización Vial Ltda es una empresa que ofrece:}
	\begin{itemize}
	\item Mantención de Señales viales.
	\item Instalación de Postes.
	\item Instalación de Señal.
	\insertimage[]{serviplott/mixproducto1}{width=6.5cm}{Señalética.}
	\item Vallas peatonales.
	\insertimage[]{serviplott/mixproducto2}{width=6.5cm}{Valla peatonal.}
	\item Demarcación de Reservados. 
	\item Conos de PVC.
	\insertimage[]{serviplott/mixproducto3}{width=4.5cm}{Cono PVC.}
	\end{itemize}
	
	\vspace{5mm}
	\noindent{\textbf{Mantención de Señales Viales:}}\vspace{3mm} 
	\noindent{Este mix consta de 10 partes que se pueden realizar en una señal ya existente, la cual puede ir desde:}\vspace{3mm}
	\begin{itemize}
	\item Enderezar o aplomar poste. 
	\item Instalación de placa Señal. 
	\item Pintado de poste.
	\item Limpieza de señal.
	\item Cambio de Poste.
	\item Retiro solo placa.
	\item Anclaje de poste.
	\item Traslado de Señal completa.
	\item Reinstalación de Señal completa.
	\item Retiro de Señal completa (cabe destacar esta última debido a que en algunas comunas y en algunas señales tiene cobro). 
	\end{itemize}
	
	\vspace{5mm}
	\noindent{\textbf{Instalación de poste: }}\vspace{3mm}
	\begin{itemize}
	\item Poste Omega de 3.0 con base
	\item Poste Omega con extensión
	\item Poste Omega 2.0
	\item Poste Omega
	\item Poste Omega 3.0
	\item Poste Omega 3.5
	\item Poste 50x50 con base
	\item Poste 50x50
	\end{itemize}
	
	\newpage
	\noindent{\textbf{Instalación de Señales:}}\vspace{3mm}
	\noindent{Para las señales existen distintos tipos de dimensiones y en algunas ocasiones figuras, tal como muestra la siguiente tabla:}\vspace{3mm}
	\insertimage[]{serviplott/dimensiones}{width=15cm}{Dimensiones de Señales.}\vspace{3mm}
	\noindent{\textbf{Vallas peatonales: }\vspace{3mm}
	\noindent{Dentro de este producto se divide en 4 trabajos que se pueden realizar sobre las vallas peatonales:}\vspace{3mm}
	\begin{itemize}
	\item Pintura de vallas peatonales.
	\item Reparación de vallas peatonales.
	\item Instalación de vallas peatonales.
	\item Provisión e instalación de vallas peatonales.
	\end{itemize}
	
	\vspace{3mm}
	\noindent{\textbf{Demarcación de Reservados: }}
	\noindent{La demarcación viene a ser la pintura que se coloca en los distintos espacios para reservados que sean solicitados ya sea por espacios públicos como demarcación de estacionamientos para personas con capacidad reducida, o espacios solicitado de privados a la municipalidades como puede ser el espacio de taxis y colectivos.}
	
	\newpage
	\subsection{Procesos internos}
	\noindent{La siguiente figura (Procedimiento de ejecucion de proyectos) es un cuadro de los procedimientos que se hacen dentro de la empresa para la ejecución de proyectos.}
	
	\insertimage[]{serviplott/procedimientoejecuciondeproyectos}{width=16cm}{Procedimiento de ejecucion de proyectos}
	\noindent{Esta figura referencia todos los procesos de la empresa serviplott en tanto a logística}
	\insertimage[]{serviplott/procesologistica}{width=14cm}{Procesos de logística}
	
\section{Analisis de Proceso de gestion financiera}

	\subsection{Identificación de Inputs}
	
	\noindent{Los inputs para el proceso financiero de Serviplott son los que se muestran a continuación:}
	
	\subsubsection{Inputs}
	
	\noindent{Materiales, equipamiento, información, recursos financieros,recursos humanos o condiciones medio ambientales requeridas para llevar a cabo el proceso financiero.}\vspace{3mm}
	
	\noindent{Proveedores: Personas (clientes) que abastecen al proceso con sus inputs.Estos pueden ser:}
	
	\begin{itemize}
	\item Depósitos bancarios.
	\item Cheques.
	\item Transferencias electrónicas.
	\item Vales vista.
	\item Otros.(Aunque no me gusta la idea de colocar etc u otros, pero en el otro informe aparecía).
	\end{itemize}
	
	\noindent{Y sus facturas correspondientes son elevadas al término de las fases del pago de pie por un proyecto y la finalización de este.}\\\\
	\noindent{Otros de los inputs que se logran identificar y que son altamente involucrados en la gestión financiera, son los siguientes:}.
	
	\begin{itemize}
	\item Boleta de servicios básicos. 
	\item Facturación.
	\item Formulario de Impuestos.
	\item Boleta de Honorarios. 
	\item Boleta de Pago.
	\item Boleta de honorarios de diseño. 
	\item Boleta de pago de dominio.
	\item Boleta de pago de hosting.
	\item Boleta de pago de artículos de oficina.
	\item Entre otros.(Aunque no me gusta la idea de colocar etc u otros, pero en el otro informe aparecía).
	\end{itemize}
	
	\subsection{Descripción Actividades}
	
	\subsubsection{Proceso de gestión financiera}
	
	\noindent{En cada una de las etapas del proceso de gestión financiera se realizan las siguientes actividades:}
	
	\begin{itemize}
	\item Recopilacion de Informacion: En esta etapa se recopila toda información necesaria o útil para las actividades del negocio, tales como, hechos económicos, información de mercado, hechos sociales, hechos ambientales, entre otros.
	\item Análisis de Requisitos​: Etapa en donde se procede a clasificar y analizar la información, de acuerdo a las políticas propias de la empresa, registrando dicha información debidamente. Además se pretende definir los requerimientos futuros de la empresa para su funcionamiento tomando en cuenta los objetivos a los que apunta.
	\item Definición de presupuesto​: Etapa en la cual se busca definir el presupuesto disponible para las actividades de la empresa y su correspondiente asignación en base a la información analizada. Además se fijan las políticas de la empresa sobre el manejo y uso de los presupuestos estimados.
	\item Contabilidad​: Etapa que busca definir los costos de las operaciones del periodo actual, además de identificar los elementos de costo que afectan las actividades de la empresa.
	\item Fijación de precios​: Etapa en la que se establecen las tarifas por productos y servicios prestados a los clientes. Para el caso de tratarse de un servicio especial el precio de este está fuertemente ligado al valor de las monedas.
	\item Control y seguimiento: Etapa que busca controlar las reglas de operación de los programas definidos y mantener un seguimiento a través de informes de resultados y el monitoreo de indicadores, planteando planes de monitoreo para validar los indicadores e informes que se han presentado.
	\item Resultados​: Etapa en donde se publican los resultados e informes de las actividades desarrolladas por la empresa.
	\end{itemize}
	
	\noindent{\textbf{Indicadores}}
	\\\\
	\noindent{\textbf{Indicadores de liquidez}}
	\vspace{3mm}
	
	\noindent{La liquidez de una organización es juzgada por la capacidad para saldar las obligaciones a corto plazo que se han adquirido a medida que éstas se vencen. Se refieren no solamente a las finanzas totales de la empresa, sino a su habilidad para convertir en efectivo determinados activos y pasivos corrientes.}
	
	\insertimage[]{serviplott/indicadorliquidez}{width=6cm}{Indicador de liquidez}
	
	\noindent{\textbf{Indicadores de eficiencia}}
	\vspace{3mm}
	
	\noindent{Establecen la relación entre los costos de los insumos y los productos de proceso; determinan la productividad con la cual se administran los recursos, para la obtención de los resultados del proceso y el cumplimiento de los objetivos. Los indicadores de eficiencia miden el nivel de ejecución del proceso, se concentran en el Cómo se hicieron las cosas y miden el rendimiento de los recursos utilizados por un proceso. Tienen que ver con la productividad.}
	
	\noindent{\textbf{Indicadores de desempeño}}
	\vspace{3mm}
	
	\noindent{Es un instrumento de medición de las principales variables asociadas al cumplimiento de los objetivos y que a su vez constituyen una expresión cuantitativa y/o cualitativa de lo que se pretende alcanzar con un objetivo específico establecido.}
	
	\noindent{\textbf{Indicadores de productividad}}
	\vspace{3mm}
	
	\noindent{La productividad está asociada a la mayor producción por cada hombre dentro de la empresa y al manejo razonable de la eficiencia y la eficacia.}
	
	\noindent{\textbf{Indicadores de endeudamiento}}
	\vspace{3mm}
	
	\noindent{Tienen por objeto medir en qué grado y de qué forma participan los acreedores dentro del financiamiento de la empresa. De la misma manera se trata de establecer el riesgo que incurren tales acreedores, el riesgo de los dueños y la conveniencia o inconveniencia de un determinado nivel de endeudamiento para la empresa.}
	
	\noindent{\textbf{Indicadores de diagnóstico financiero}}
	\vspace{3mm}
	
	\noindent{El diagnóstico financiero es un conjunto de indicadores qué, a diferencia de los indicadores de análisis financiero, se construyen no solamente a partir de las cuentas del Balance General sino además de cuentas del Estado de Resultados, Flujo de Caja y de otras fuentes externos de valoración de mercado. Esto conlleva a que sus conclusiones y análisis midan en términos más dinámicos, y no estáticos, el comportamiento de una organización en términos de rentabilidad y efectividad en el uso de sus recursos.}
	
	\noindent{\underline{Humanware}}
	\vspace{3mm}
	
	\begin{itemize}
	\item Gerencia: Persona o grupo de personas encargadas de la toma de decisiones de la empresa, además de ser quienes definen las políticas que dirigen la misma.
	\item Contador General​: Persona encargada de llevar a cabo la contabilidad de la empresa y que entre sus funciones está, llevar el control sobre las distintas actividades que constituyen el movimiento contable y que dan lugar a los balances y demás reportes financieros.
	\item Contador-Auditor​ : El contador auditor es aquella persona que verifica y certifica la información generada por el contador general, esto en busca de posibles errores y “fugas” que se puedan estar generando y no estén siendo incluidas en los informes del contador general. Este Contador, al tener mayor responsabilidad y peso, también tiene un nivel de jerarquía alto, pudiendo dirigirse directamente a cualquier empleado y sistema de información de la empresa para validar los datos.
	\item Auditor Externo​ : Este Auditor (que generalmente es un contador-auditor en el caso financiero) es el encargado de verificar y certificar que la información contenida en los informes de gestión financiera sean los correctos y se adapten a los requeridos por las normas vigentes. Generalmente este auditor es contratado para una auditoría a final de año dirigida al director, socios y accionistas, aunque también se le puede solicitar en casos específicos de pérdidas de dinero por robo, fraude o negligencia. 
	\end{itemize}
	
	\noindent{\underline{Software}}
	\vspace{3mm}
	
	\noindent{Los principales elementos de software utilizados para apoyar el proceso son: }
	
	\begin{itemize}
	\item MS-Office: Paquete de softwares de oficina, contiene el popular programa de ofimática “Microsoft Excel”, ampliamente utilizado por contadores para presentacion de informacion en forma de Tabla.
	\item SAP: EuroAmerica procesa toda su información financiera, productiva y comercial a través de SAP. Este ERP mantiene un registro y control de acceso a información a nivel de consulta y proceso, mediante la asignación de perfiles de usuarios.
	\item SQL: Sistema de Base de Datos utilizado para contener información de todos los ámbitos de la empresa, entre ellos, financieros.
	\end{itemize}
	
	\noindent{\underline{Hardware}}
	\vspace{3mm}
	
	\noindent{Los principales hardware utilizados para apoyar el proceso son:}
	
	\begin{itemize}
	\item Computadoras personales (Notebooks y PC): Estos son provistos por la empresa a sus empleados y son ampliamente utilizados para el cálculo, el almacenamiento y la comunicación que sirve para los procesos financieros.
	\item Servidores: Estos están alojados dentro y fuera de la empresa y su contenido es manejado por el administrador de servicios TI, contienen información general y específica de la empresa, entre ella información de ventas y compras, que apoyan el proceso de gestión financiera.
	\end{itemize}
	
	\subsection{Descripción de Outputs y sus Características}
	
	\subsubsection{Output}
	
	\noindent{El producto o servicio creado por el proceso; el que se entrega al cliente. En este caso, son los servicios que presta la empresa, como también el estado actual de cada proceso, como son los estados financieros, estadísticas de la empresa, memorias, entre otras cosas con relación a la empresa.}\\\\
	\noindent{Cliente: Los que utilizan su output. Tanto si sus clientes son internos como externos, utilizan el output como un input para sus procesos de trabajo.}\\\\
	\noindent{Los outputs que el proceso financiero suelen ser presentados en formato de informe, los más destacados de estos son:}
}
	
	\subsection{Caracterización del cliente}
	\subsection{Análisis de control de calidad}
	
	\noindent{La empresa EuroAmerica, tiene sus sitemas de control de calidad, los cuales fueron mencionados anteriormente, juntos con sus riesgos, comités relacionados etc, Esta cuenta con auditorias tanto internas, como externas.}
	
	\subsubsection{Auditorías constantes Internas}
	\subsection{Identificación y caracterización de las TIC de apoyo}
	
\section{Propuesta de PA de Mejora}
\subsection{Actividades, tiempos, responsables, recursos y Control}
\section{Conclusión}
\section{Bibliografia}

	

