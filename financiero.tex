% Template:     Informe/Reporte LaTeX
% Documento:    Archivo de ejemplo
% Versión:      5.5.4 (16/09/2018)
% Codificación: UTF-8
%
% Autor: Pablo Pizarro R. @ppizarror
%        Facultad de Ciencias Físicas y Matemáticas
%        Universidad de Chile
%        pablo.pizarro@ing.uchile.cl, ppizarror.com
%
% Manual template: [http://latex.ppizarror.com/Template-Informe/]
% Licencia MIT:    [https://opensource.org/licenses/MIT/]

\section{Introducción}
\noindent{Día a día el parque automotriz aumenta y es necesario mantener y tener nuevas señaléticas. Debido a esta necesidad es que nace Serviplott, una empresa la cual se encarga de vender servicios a distintos municipios los cuales requieren nuevas señaléticas y mantención de las señales viales ya existentes para así mejorar la experiencia de conducción.}\vspace{5mm}

\noindent{Toda municipalidad tiene como tarea manejar las señales viales existentes en su sector. Al tener un número alto de señales en los distintos sectores es que las municipalidades optan por externalizar este servicio y se lo dejan a empresas importantes tales como Serviplott.}\vspace{5mm}

\noindent{En el presente trabajo se desarrollará un completo análisis a la gestión de los procesos Industriales de Serviplott, identificando todos los problemas asociados a esta área, para luego proponer un plan de mejora en el ámbito de la Gestión de Procesos.}\vspace{5mm}

\insertimage[]{serviplott/logo2}{width=6cm}{Logo Empresa.}\newpage

\section{Objetivos}
	\subsection{Objetivo General}
	\noindent{El objetivo general consiste en investigar el proceso de gestión financiera llevado a cabo por la empresa “Serviplott Servicios Gráficos y Señalización Vial Ltda” además de analizar como esta apoya sus procesos con el uso de las TI (Tecnologías de la información).}
	\subsection{Objetivos Específicos}
	\begin{itemize}
	\item Analizar los datos obtenidos de la organización, con la finalidad de elaborar una opinión crítica y atingente sobre los diferentes aspectos que conforman el proceso de gestión financiera de Serviplott.
	\item Analizar los datos obtenidos de la organización, con la finalidad de elaborar un plan de acción de mejoras.
	\end{itemize}
	
\section{Antecendetes Generales de la Empresa}

	\subsection{Identificación de la Empresa}
	\noindent{Fundada en 2002, Serviplott es una empresa conformada por profesionales del área de la ingeniería mecánica y la ingeniería en transporte. Estas características le dan a la empresa flexibilidad, rapidez y soporte técnico que demanda el mercado de la infraestructura vial. En Serviplott creen en una relación personalizada con sus clientes y por ello ofrecen tiempos de respuesta de acuerdo a sus particulares necesidades con altos estándares de calidad.}\vspace{5mm}

	\noindent{Inicialmente se encargaban de generar gráficos publicitarias y seguridad, también entregando proyectos de ingeniería en tránsito. Con el tiempo serviplott comenzó a especializarse solamente en soluciones de ingeniería en tránsito, donde se especializó en señales viales.}\vspace{5mm}

	\noindent{A partir de ello, serviplott comienza a licitar en las distintas comunas para así ser una empresa de renombre en mantención e instalación de señaléticas viales. Hoy en día serviplott se encuentra encargado de comunas tales como Santiago, Lo Barnechea, Macul, Ñuñoa.}\vspace{5mm}

	\noindent{Serviplott como empresa se encuentra en una constante actualización de técnicas y tecnologías capacitando a sus distintos trabajadores  en las distintas áreas.}

	\subsection{Misión}
	\noindent{Serviplott es una empresa conformada por profesionales del área de la ingeniería mecánica y la Ingeniería en transportes. Estas características le dan a la empresa la flexibilidad, rapidez y soporte técnico que demanda el mercado de la infraestructura vial. En Servilplott creemos en una relación personalizada con nuestros clientes y por ello ofrecemos tiempos de respuesta de acuerdo a sus particulares necesidades con altos estándares de calidad.}
	
	\subsection{Visión}
	\noindent{Como el mejor proveedor de productos y servicios viales, siendo reconocidos por el mercado como una empresa seria, profesionales y confiable orientada a la satisfacción de sus clientes y de los usuarios del servicio.}
	
	\subsection{Valores}
	\noindent{Nos importa el bienestar de nuestro entorno y colaboradores, preocupándonos de mantener niveles óptimos de seguridad y cumpliendo con las exigencias de las industrias que atendemos. Los materiales y productos utilizados en nuestras obras provienen de proveedores con trayectoria a nivel mundial. Por otro lado nos aseguramos de entregar un servicio de mantención y post venta con respuestas ágiles y eficaces a nuestros clientes.}\vspace{5mm}
	
	\noindent{Es nuestra responsabilidad cumplir con esta promesa y el mayor reconocimiento está en la decisión de nuestros clientes de preferir contratar nuestros servicios. Somos más que energía térmica: estamos comprometidos con el desarrollo y trabajo bien hecho.}
	
	\subsection{Organigrama}
	\noindent{La siguiente figura es una representación gráfica de la empresa, en el cual se muestran las relaciones de jerarquías entre sus diferentes partes.}\vspace{5mm}
	
	\noindent{Este organigrama fue entregado por la empresa que se está analizando, y es un modeleo abstracto y sistemático que permite obtener una idea uniforme y sintética de la estructura formal de la empresa Serviplott.}
	\insertimage[]{serviplott/organigrama}{width=11.2cm}{Organigrama Empresa.}
	
	\subsubsection{Descripción de cargos}
	
	\begin{enumerate}
	\item CEO: Es el director general de la empresa sobre el cual caen todas las responsabilidades.
	\item Gerente General : Es el encargado de coordinar los elementos Finanzas, Producción, Logística y Comercial.
	\item Finanzas
	  \begin{enumerate}
	  \item Contabilidad : Esta área es encarga de emitir facturas y llevar registro de todos los procesos contables de la empresa.
	  \item Cobranza : Existen cobradores encargados de ir a buscar los pagos y de llamar a clientes con deuda vigente.
	  \end{enumerate}
	\item Producción
	  \begin{enumerate}
	    \item Técnicos de producción : Empleados encargados de producir los pedidos.
	    \item Técnicos en terreno : Personal que acude a terreno para montar el pedido.
	  \end{enumerate}
	    \item Logística
	  \begin{enumerate}
	    \item Bodegueros : Encargados de llevar registro del inventario asi como entradas y salidas de la bodega. También genera una solicitud de materias primas cuando están en un bajo nivel.
	    \item Transportistas : Mueven los insumos desde los proveedores hasta la bodega y movilizan a los técnicos en terreno.
	  \end{enumerate}
	\item Comercial
	  \begin{enumerate}
	    \item Ventas : Personal encargado de gestionar las ventas que se hacen.
	    \item Marketing : Área encarga del marketing para promocionar la empresa por distintos medios.\newpage
	  \end{enumerate}
	\end{enumerate}
	
	\subsection{Mix de Productos}
	\noindent{Serviplott Servicios Gráficos y Señalización Vial Ltda es una empresa que ofrece:}
	\begin{itemize}
	\item Mantención de Señales viales.
	\item Instalación de Postes.
	\item Instalación de Señal.
	\insertimage[]{serviplott/mixproducto1}{width=6.5cm}{Señalética.}
	\item Vallas peatonales.
	\insertimage[]{serviplott/mixproducto2}{width=6.5cm}{Valla peatonal.}
	\item Demarcación de Reservados. 
	\item Conos de PVC.
	\insertimage[]{serviplott/mixproducto3}{width=4.5cm}{Cono PVC.}
	\end{itemize}
	
	\vspace{5mm}
	\noindent{\textbf{Mantención de Señales Viales:}}\vspace{3mm} 
	\noindent{Este mix consta de 10 partes que se pueden realizar en una señal ya existente, la cual puede ir desde:}\vspace{3mm}
	\begin{itemize}
	\item Enderezar o aplomar poste. 
	\item Instalación de placa Señal. 
	\item Pintado de poste.
	\item Limpieza de señal.
	\item Cambio de Poste.
	\item Retiro solo placa.
	\item Anclaje de poste.
	\item Traslado de Señal completa.
	\item Reinstalación de Señal completa.
	\item Retiro de Señal completa (cabe destacar esta última debido a que en algunas comunas y en algunas señales tiene cobro). 
	\end{itemize}
	
	\vspace{5mm}
	\noindent{\textbf{Instalación de poste: }}\vspace{3mm}
	\begin{itemize}
	\item Poste Omega de 3.0 con base
	\item Poste Omega con extensión
	\item Poste Omega 2.0
	\item Poste Omega
	\item Poste Omega 3.0
	\item Poste Omega 3.5
	\item Poste 50x50 con base
	\item Poste 50x50
	\end{itemize}
	
	\newpage
	\noindent{\textbf{Instalación de Señales:}}\vspace{3mm}
	\noindent{Para las señales existen distintos tipos de dimensiones y en algunas ocasiones figuras, tal como muestra la siguiente tabla:}\vspace{3mm}
	\insertimage[]{serviplott/dimensiones}{width=15cm}{Dimensiones de Señales.}\vspace{3mm}
	\noindent{\textbf{Vallas peatonales: }\vspace{3mm}
	\noindent{Dentro de este producto se divide en 4 trabajos que se pueden realizar sobre las vallas peatonales:}\vspace{3mm}
	\begin{itemize}
	\item Pintura de vallas peatonales.
	\item Reparación de vallas peatonales.
	\item Instalación de vallas peatonales.
	\item Provisión e instalación de vallas peatonales.
	\end{itemize}
	
	\vspace{3mm}
	\noindent{\textbf{Demarcación de Reservados: }}
	\noindent{La demarcación viene a ser la pintura que se coloca en los distintos espacios para reservados que sean solicitados ya sea por espacios públicos como demarcación de estacionamientos para personas con capacidad reducida, o espacios solicitado de privados a la municipalidades como puede ser el espacio de taxis y colectivos.}
	
	\newpage
	\subsection{Procesos internos}
	\noindent{La siguiente figura (Procedimiento de ejecucion de proyectos) es un cuadro de los procedimientos que se hacen dentro de la empresa para la ejecución de proyectos.}
	
	\insertimage[]{serviplott/procedimientoejecuciondeproyectos}{width=16cm}{Procedimiento de ejecucion de proyectos}
	\noindent{Esta figura referencia todos los procesos de la empresa serviplott en tanto a logística}
	\insertimage[]{serviplott/procesologistica}{width=14cm}{Procesos de logística}
\section{Analisis de Proceso de gestion financiera}
	\subsection{Identificación de Inputs}
	\subsection{Descripción Actividades}
	\subsection{Descripción de Outputs y sus Características}
	\subsection{Caracterización del cliente}
	\subsection{Análisis de control de calidad}
	\subsubsection{Auditorías constantes Internas}
	\subsection{Identificación y caracterización de las TIC de apoyo}
	
\section{Propuesta de PA de Mejora}
\subsection{Actividades, tiempos, responsables, recursos y Control}
\section{Conclusión}
\section{Bibliografia}

	

